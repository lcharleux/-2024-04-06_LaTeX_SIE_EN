\documentclass[10pt,a4paper]{article}
\usepackage[utf8]{inputenc}
\usepackage[english]{babel}
\usepackage{amsmath}
\usepackage{amsfonts}
\usepackage{amssymb}
\usepackage{blindtext}
\usepackage[left=2cm,right=2cm,top=2cm,bottom=2cm]{geometry}

\title{My first \LaTeX~ document}
\author{SIE doctoral school}
%\date{June 2023}
\date{\today}

\begin{document}
\maketitle % THIS COMMAND WILL CREATE THE TITLE

\section{Introduction}

In this article, we will  deal with basic stuff about \LaTeX.

\tableofcontents



\section{Basic stuff}

You can use \% this way \& with the backslash \textbackslash.
You can create a new \\ line.

\vspace{2cm}

Hello again !

\section{Things about math}

This is dumb text to create a paragraph.
You can write inline math $\alpha = 36$.
\blindtext[1]
It is also possible  to do standalone math this way:

$$
    u(x) = \int_{t=0}^{+\infty} g_{ij}^2(t,x) dt
$$

\noindent With: % Here I don't want to have an indent

\begin{enumerate}
    \item Firstly: $g_{ij} = 5$
    \item And also: $x \in \; ] 0, +\infty [$ % "\;" is a small space 
\end{enumerate}

\noindent You can use numbered equations like this very interresting Eq. \ref{eq:useless}.

\begin{equation}
    z = \sum_{n=0}^N v_n
    \label{eq:useless} % this is the label of the equation.
\end{equation}

\begin{eqnarray}
    a =  5 \nonumber \\ % THIS ONE WILL HAVE NO NUMBER
    b = 12 \\
    0 = 0
\end{eqnarray}

\begin{equation}
    f(x) =
    \left\lbrace
    \begin{split}
        x \mbox{ if : } x > 0 \\
        0 \mbox { otherwise }
    \end{split}
    \right.
\end{equation}


\end{document}